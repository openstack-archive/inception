\section{Overview}
Creating an inception cloud consists of preparing your workstation, preparing the VM environment by adding a temporary boot-up machine, 
and then executing the \verb!orchestrator! programme to build the cloud.  
Once the inception cloud has been created, there is a small amount of housekeeping that is necessary to be able to reach the new cloud. 
This document describes the software necessary to support and manage an inception cloud, lists the steps needed to start an 
inception cloud, provides information on accessing the inception cloud and includes the command needed to stop the cloud. 
The final section contains a small bit of guidance which addresses how to access VMs running on the inception cloud.

Some knowledge of OpenStack, virtual machines and the Nova client command line interface is assumed. 
This document provides some OpenStack commands to assist with VM setup, but makes no attempt to explain 
their syntax or to further illustrate any command line parameters that are not directly used in the examples. 

\subsection{Workstation Requirements }
Typically, an inception cloud is started and managed from a regular user account on a remote workstation. 
There are several software requirements that are necessary for the workstation, and the user must have slightly 
advanced privledges (sudo).
The following list the workstation and user requirements:

\begin{itemize}
\item User must have \verb!sudo! privileges that allow /etc/hosts and iptables to be modified
\item Python version 2.7
\item IPython version 0.13.2 or later
\item Sshuttle 
\item Nova Client version 2.13.0 or later
\item Nova oslo.config version 1.1.1 or later
\item Git 
\item Inception software from github
\end{itemize}

\noindent
It is possible to use the boot-up VM, described later,  as the workstation instead of using it just as a passthrough machine.  
With the exception of when sshuttle is needed, the setup and startup of an inception cloud is nearly the 
same regardless of whether the "workstation" is real or virtual. 
This document will address just the use of a physical workstation, 
outside of the OpenStack virtual world, to create and manage an inception cloud; the reader is left to make the 
small extrapolations that allow the boot-up VM to be used in place of a real workstation if that is desired.
