\section{Starting and Using iVMs}
The dashboard and\/or nova command line commands can be used to start and manage iVMs within the inception environment.  
The iVMs are allocated across the worker ICVM(s) that were created by orchestrator.
The nova command line interface can be used on either the controller ICVM (Ct in the illustration), or directly from 
the workstation (provided that the proper environment variables have been defined and that sshuttle is tunneling traffic
to the private control network).  
Figure \ref{fg_planes} illustrates the logical relationship of the inception \emph{plane} with the hosting 
OpenStack (virtual plane) environment. 


\begin{figure}[!ht]
  \centering
  \includegraphics[width=\linewidth]{inception_planes.pdf}
  \tiny
  \caption{\small The relationship between the inception plane and the virtual plane.
    \label{fg_planes}\normalsize
  }
  \normalsize
\end{figure}

There are two methods which can be used to ssh into an iVM host. 
First is to \verb!ssh! into the gateway, then to \verb!ssh! to a worker ICVM, and finally to \verb!ssh! to the desired iVM. 
The second is to \emph{string} all of the ssh commands together into a single command.  
Regardless of which method is used, the \verb!-A! and \verb!-i! command line options will need to be given on the initial
ssh command in order to forward authorisation with each step, and to use the private key that is associated with the ICVMs. 
While it is necessary to use the \verb!-i! option only on the first command, the \verb!-A! option must be given on all of the
ssh commands.  

If a single command line is used, it will also be necessary to use the \verb!-t! option to force ssh to treat each \emph{hop} as 
a terminal session.  
The next command example illustrates a single ssh command that creates a log-in session on an iVM.

% this is silly -- latex has no conditional 'eject' and so we must wrap in a tabular to ensure it does not split
% if there isn't room for the whole three lines on the current page/column. 
\small\begin{tabular}{p{\textwidth}}\begin{verbatim}
   ssh -t -A -i agave.shared.pk ubuntu@135.207.223.158  \
       ssh -t -A ubuntu@scooter0-worker1  \
       ssh -t ubuntu@10.252.0.2
\end{verbatim}
\end{tabular}
\normalsize


If this command is going to be executed from a script, or just to make life easier, it  might be good to add the following options
to each of the ssh commands.

\small\begin{verbatim}
   -o UserKnownHostsFile=/dev/null -o StrictHostKeyChecking=no 
\end{verbatim}\normalsize

These prevent information about the hosts being accessed  from being saved in the \verb!known_hosts! file and prevent ssh from 
complaining if the host information was previously saved and is different from the current information. 
Be aware that because the host information is being pushed to \verb!/dev/null,! ssh will indicate that it has been added 
each time the commands are executed (this is less bothersome than having to clean out the known hosts file as the 
VM host information changes).
