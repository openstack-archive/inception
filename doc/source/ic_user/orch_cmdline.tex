% command line options appendix

\section{Orchestrator Options}
The following is a complete list of the command line options supported by \verb!orchestrator.!\footnote{Because 
some, but not all, text formatters assume two consecutive dash characters (-) are to be 
treated as a long single dash, options which require two leading dashes are depicted in this section with 
two dashes separated by a space in order to emphasise that there are two dashes; the space 
must be omitted when enterfing these options on the command line.}

\dlbeg{1.65in}
\dlitem{-~-atomic}{ 
	When supplied causes an automatic roll back (delete all virtual resources) when an error during
	startup is encountered.  
}

\vspace{5pt}
\dlitem{-~-chef\_repo=URL}{ 
	Supplies the URL of the Chef git repository. If not supplied 'git://github.com/att/inception-chef-repo.git' is used. 
}

\vspace{5pt}
\dlitem{-~-chef\_repo\_branch=name}{ 
	Lists the branch name within the Chef git repository to use when fetching Chef.  If not supplied 'master' is used. 
}

\vspace{5pt}
\dlitem{-~-chef\_server\_image=name}{ 
	Supplies the name of the OS virtual machine image that should be used when creating the Chef server ICVM. 
	As a default, 'u1204-130716-gvc' is used. 
}

\vspace{5pt}
\dlitem{-~-dst\_dir=name}{ 
	The absolute path of the directory where Chef related setup scripts are to be placed on the ICVMs. 
	When not defined, /home/ubuntu is used. 
}

\vspace{5pt}
\dlitem{-~-flavor=name}{ 
	Indicates the name of the VM flavour that will be used for ICVMs. If not supplied the model m1.tiny is used. 
}

\vspace{5pt}
\dlitem{-~-gateway\_flavor=name}{ 
	Indicates the name of the VM flavour that will be used for the gateway ICVMs. If not supplied the model m1.tiny is used. 
}

\vspace{5pt}
\dlitem{-~-image=name}{ 
	Supplied the image name to be used for all ICVMs.  If not supplied the name u1204-130621-gv is used for each ICVM. 
}

\vspace{5pt}
\dlitem{-~-ssh\_keyfile=path}{ 
	Provides the path to the private key that is to be injected as the user key for each of the 
	control VMs that are created.
}

\vspace{5pt}
\dlitem{-~-key\_name=name}{ 
	Supplies  the name of the key file that is to be used for ssh access to the ICVMs. If not supplied the 
	name 'shared' is used. 

}

\dlitem{-p name}{ }
\dlitem{-~-prefix=name}{ 
	This command line flag is required and supplies the prefix string that is used when defining the ICVM names. 
	(The name given may not contain hyphens.)
}

\vspace{5pt}
\dlitem{-~-parallel}{ 
	Causes Chef related setup tasks to be executed in parallel. 
}

\vspace{5pt}
\dlitem{-~-poll\_interval=n}{ 
	Specifies the number of seconds between retries when testing for the state and stability of the ICVMs. 
	When not supplied, 5 seconds is used. 
}

\vspace{5pt}
\dlitem{-n n}{ }
\dlitem{- -num\_workers=n} {
	Specifies the number of work ICVMs that are created.  The iVMs which are created in the inception cloud are hosted 
	on the work VMs thus the number needed is directly related to the number of iVMs that will be created in the inception cloud. 
}

\vspace{5pt}
\dlitem{-~-pool=name}{ 
	Defines the name of the pool from which floating IP addresses will be taken. If not supplied the 
	name 'research' is used. 
}

\vspace{5pt}
\dlitem{-~-sdn}{ 
	When supplied causes an SDN/OpenFlow and full-mesh network topology to be used. If not supplied
	a conventional L2/L3 star topology is used. 
}

\vspace{5pt}
\dlitem{-~-security\_groups=list}{ 
	Supplies a comma separated list of security group names (already known to the hosting VM environment) 
	that are to be assigned to each ICVM created. If not supplied default and ssh are used. 
}

\vspace{5pt}
\dlitem{-~-src\_dir}{ 
	the relative path to the directory containing Chef related setup scripts on the current host. 
	If not supplied ../bin is assumed. 
}

\vspace{5pt}
\dlitem{- -user=name}{ 
	The user name created on each node with sudo capabilities. If not given, ubuntu is used. 
}

\vspace{5pt}
\dlitem{-~-user\_data=string}{ 
	Provides a bash script that is run by cloud-init during the boot stage (late).	
}

\vspace{5pt}
\dlitem{-~-timeout=n}{ 
	Sets the number of seconds, default unlimited (nearly), to wait for all ICVMs to become ready.
}
\dlend

